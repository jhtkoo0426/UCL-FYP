\documentclass[12pt, a4paper]{report}
\usepackage{setspace}
%\usepackage{subfigure}

% Page styling
%\pagestyle{plain}
\usepackage{amssymb,graphicx,color}
\usepackage{amsfonts}
\usepackage{extsizes}
\usepackage{latexsym}
\usepackage[a4paper,margin=3cm]{geometry}
\usepackage{changepage}
\usepackage{mathptmx}

% Tables
\usepackage{multirow, multicol}
\setlength{\columnsep}{1cm}

% Custom theorems
\usepackage[font=small, labelfont=bf]{caption}
\usepackage[protrusion=true, expansion=true]{microtype}
\usepackage{sectsty}
\usepackage{url, lipsum}
\usepackage{amsthm}

\usepackage{hyperref}

\theoremstyle{definition}
\newtheorem{theorem}{THEOREM}
\newtheorem{lemma}[theorem]{LEMMA}
\newtheorem{corollary}[theorem]{COROLLARY}
\newtheorem{proposition}[theorem]{PROPOSITION}
\newtheorem{remark}[theorem]{REMARK}
\newtheorem{definition}[theorem]{DEFINITION}
\newtheorem{fact}[theorem]{FACT}
\newtheorem{example}[theorem]{EXAMPLE}
\newtheorem{property}[theorem]{PROPERTY}
\def \set#1{\{#1\} }

% Positioning figures
\usepackage{float}

% Equations
\usepackage{amsmath}

% Code
\usepackage{fancyvrb}

% Spacing
\newcommand{\HRule}[1]{\rule{\linewidth}{#1}}
\setcounter{tocdepth}{5}
\setcounter{secnumdepth}{5}

% Helper packages - delete these before submission
\usepackage{xcolor}


%----------------------------------------------------------------------------------
% Title Page
%----------------------------------------------------------------------------------
\title{{\vspace{-2in}\includegraphics[scale=0.4]{docs/Project Report/Media/ucl_logo.png}}\\
\vspace{2cm}
\begin{adjustwidth}{1cm}{1cm}
    \centering
    \Huge Robust Robotic Grasping Utilising Touch Sensing
\end{adjustwidth}}
\\
\date{Submission date: \today}
\author{Koo Ho Tin Justin\thanks{
{\bf Disclaimer:}
This report is submitted as part requirement for the BSc degree in Computer Science at UCL. It is
substantially the result of my own work except where explicitly indicated in the text.
\emph{Either:} The report may be freely copied and distributed provided the source is explicitly acknowledged
\newline  %% \\ messes it up
\emph{Or:}\newline
The report will be distributed to the internal and external examiners, but thereafter may not be copied or distributed except with permission from the author.}
\\ \\
BSc Computer Science\\ \\
Supervisors: Prof. Marc Deisenroth, Dr. Yasemin Bekiroglu}



\begin{document}
\onehalfspacing
\maketitle


%----------------------------------------------------------------------------------
% Abstract
%----------------------------------------------------------------------------------
\begin{abstract}
Robotic grasping is a fundamental subdomain of robotics that involves the manipulation of objects using a gripper or end effector. Many different approaches have been proposed for robotic grasping, with various assumptions regarding the available information about the scene (robot setup, environment) and the type of objects in question (known, unknown, familiar). These approaches range from designing simplifying hand models, to completely end-to-end systems inferring grasp parameters from raw data using generative models.\\

However, the majority of these approaches do not address robustness in grasping, which refers to the ability of a robot to perform a grasping task consistently and accurately despite changes in the object.\\

This project addresses several fundamental problems for achieving better robustness in grasping tasks. These include, for example, dealing with uncertainties in sensing and actuation, scalability and adding constraints in terms of task. In this project, we present a robust learning framework that attempts to perform inferred grasp configurations from hand poses, using simple geometry on a two-finger hand. The project addresses two core complications for the framework: a) balancing the trade-off between data representation and data dimensionality; and b) a correction policy which aims to gradually synthesise good grasps.
\end{abstract}


%----------------------------------------------------------------------------------
% Acknowledgements
%----------------------------------------------------------------------------------
\renewcommand\abstractname{Acknowledgments}
\begin{abstract}
    
\end{abstract}


%----------------------------------------------------------------------------------
% Contents
%----------------------------------------------------------------------------------
\tableofcontents
\setcounter{page}{1}


%----------------------------------------------------------------------------------
% Chapter 1: Introduction
%----------------------------------------------------------------------------------
\chapter{Introduction}
\label{chap:1}

\section{Outline of Problem}
\label{sec:1.1}
The project first analyses various literature concerned with learning frameworks to robotic grasping that involves different combinations of sensory (tactile and visual) data. These frameworks mainly adapt either a learning-based approach to formulate a regrasping policy, or tactile exploration to maximise grasp quality and improve object shape representation.\\

Backed with survey findings, a simple Pybullet simulation is developed to collect tactile and visual data. This data is essential for training a logistic regression classifier to tackle a binary classification problem for good and bad labelled grasps. The classifier serves as the baseline approach for the project and a means to discover the best data representation for the data in question.\\

Finally, the project will propose a learning framework that infers hand poses for good grasps using a generative model with Gaussian processes. The framework is also tested on a real setup of the robot and the performances will be displayed and analysed in depth.


\section{Project Aims and Objectives}
\label{sec:1.2}

\subsection{Aims}
\label{sec:1.2.1}
The aim of this project is for the development and testing of a robust robotic system that can learn to pick up an object with simple geometry using a two-finger hand. The project will take a learning-based approach to grasping through, for example, Bayesian optimization \cite{nogueria, frazier}. The learning-based approach should be compared with a baseline approach from the related literature (e.g. \cite{nogueria, danielczuk, breyer}) for evaluation. 

\subsection{Objectives}
\label{sec:1.2.2}
The project aim is divided into several objectives that are expected to be completed throughout the academic year. 
\begin{enumerate}
    \item Set up a simulation environment, e.g. PyBullet or NVIDIA Isaac. 
    \item Create a data collection pipeline for sensory data (e.g. visual and force/torque readings) via the simulator.
    \item Apply basic simulation functionalities: position control and vision sensing on the robot.
    \item Implement and test baseline (e.g. \cite{breyer}, and a basic approach such as executing predefined grasps per object model given object pose).
    \item Build the learning framework:
    \begin{enumerate}
        \item Learning grasps based on Bayesian Optimization, from a chosen scene representations such as signed distance function 
        \item Picking with two fingers given object model (primitive shapes such as box, sphere, cylinder) and pose, all learned by trial and error
    \end{enumerate}
\end{enumerate}

\subsection{Additional Objectives}
These are additional objectives depending on the progress of the project:
\begin{enumerate}
    \item Quickly adapt to new object shapes based on prior trials with other shapes.
    \item Add simple touch feedback in simulation (e.g. \cite{bekiroglu}).
    \item Include task constraints: e.g. moving a cup or keeping the cup upright.
\end{enumerate}


\newpage
\section{Project Approach}
\label{sec:1.3}
We approached the project in four stages. The first stage involves the implementation of a simple Pybullet environment that simulates a \hyperref[sec:]{real robot setup} at the Statistical Machine Learning Group lab. The environment enables robot and gripper control for future stages of the project. The second stage was the development of a data collection pipeline in the simulation for various primitive object types (block, cylinder) to gather tactile and visual data. The third stage is concerned with the implementation and testing of a logistic regression classifier as the baseline approach. The final stage focuses on the implementation and testing of a learning-based approach.\\

Any technologies and tools required for the project will also be stated and justified in the corresponding sections.


\section{Report Structure}
\label{sec:1.4}
Chapter 2 covers the background and literature review for the project, including important terms and definitions, core packages for the project as well as the analysis of several related literature concerning the data representation of tactile, sensory and temporal data, robotic regrasping approaches and tactile exploration.\\

Chapter 3 documents the development of the Pybullet simulation for the project and the experimentation and analysis of physical DIGIT sensors.\\

Chapter 4 documents the development and testing of the baseline approach logistic regression classifier). It incorporates findings from Chapter 2 to discover the best data representation (raw, PCA, CNN) and the type of data to include (tactile only, visual only, both).\\

Chapter 5 documents the development and implementation of the proposed approach using a Gaussian process generative model that attempts to infer good grasp configurations from random hand poses.\\

Chapter 6 summarises the results and performances of the project, as well as making suggestions for future work and improvement based on these findings.


%----------------------------------------------------------------------------------
% Section 2: Literature Review
% Focuses on important terms used throughout the report & academic papers read in-depth
%----------------------------------------------------------------------------------
\chapter{Background and Literature Review}
\label{chap:2}

\section{Robotic Grasping}
\label{sec:2.1}
Robotic grasping refers to the ability of a robot to pick up and manipulate objects using its mechanical grippers or other end effectors. It requires precise coordination between the visual perception of the environment and efficient grasp planning, as well as robustness in terms of the object type, which is the main focus of this project.\\

There are several challenges in robotic grasping, such as object recognition, pose estimation, and grasp planning. To grasp an object, a robot needs to detect the object's location (through, for example, a wrist camera), orientation, and size accurately, estimate the best grasp point and orientation, and then execute the grasp with appropriate force and control to ensure a secure hold. The grasping process can be performed using various types of end effectors, including grippers, suction cups, or specialized tools designed for specific tasks. In this project, we will use a pair of DIGIT tactile sensors mounted onto each finger of the gripper to collect sensory readings.


\section{Pybullet}
\label{sec:2.2}
Pybullet \cite{pybullet} is a physics engine and simulator designed to be used in robotics, machine learning and computer graphics. It is a Python wrapper for the Bullet Physics Library, which is an open-source physics engine that can simulate rigid body dynamics, soft body dynamics, and fluids.


\section{Robotic Arms}
In grasping tasks, we are primarily concerned with the manipulation of a robotic arm, identify the object and guide the arm towards it to perform a grasp. A physical robotic arm consists of several important components that work together to perform various grasping tasks:
\begin{enumerate}
    \item \textbf{Base}
    \item \textbf{Joints}
    \item \textbf{Actuators}
    \item \textbf{End Effectors}
    \item \textbf{Sensors}
\end{enumerate}

These components are collectively controlled by the robot's control system, which is responsible for controlling the movement of the arm, processing sensor data, and coordinating the actions of every component. The control system may be a simple physical microcontroller, but in the context of this project, we will use the Pybullet \cite{pybullet} package to achieve our robot manipulation functions.

\section{Related Literature}
\label{sec:2.3}
\subsection{More Than a Feeling: Learning to Grasp and Regrasp using Vision and Touch (Calandra et al)\cite{calandra}}
\label{sec:2.3.1}


\subsection{Generalizing Regrasping with Supervised Policy Learning (Hausman et al)\cite{hausman}}
\label{sec:2.3.2}


\subsection{Simultaneous Tactile Exploration and Grasp Refinement for Unknown Objects (Farias et al)\cite{farias}}
\label{sec:2.3.3}


\section{Conclusion on Literature Review}
\label{sec:2.4}


%----------------------------------------------------------------------------------
% Section 3: Developing a Pybullet simulation
% Describes the detailed process of developing the simulation for the project
%----------------------------------------------------------------------------------
\chapter{Experimentation and Development of Pybullet Simulation}
\label{chap:3}
Throughout this project, we utilised the Pybullet \cite{pybullet} physics engine to develop our baseline and proposed learning models. This section provides the details for the practical aspects of the project. We will: 1) describe the robot setup which is responsible for the experimentation and validation of our approaches; 2) provide and overview for the testing and calibration of physical DIGIT \cite{digit} tactile sensors, and 3) document the development of the Pybullet simulation to collect tactile, visual and geometric data required for training our models.

\section{Robot Setup}
\label{sec:3.1}
We use a robotic arm as our robot setup in this project. The setup includes the following components:
\begin{enumerate}
    \item A Panda robot developed by the German Robotics company Franka Emika \cite{franka}, consisting of a robotic arm with an end effector. We will refer this component as the Panda arm in further sections.
    \item A 2-Finger Adaptive Robot Gripper 85 \cite{robotiq} mounted to the end effector of the Panda robot
    \item Two DIGIT \cite{digit} tactile sensors, one mounted on each finger of the gripper. These tactile sensors are responsible for providing tactile readings and feedback as depth and RGB color images.
\end{enumerate}
{\color{red}If possible add a picture with annotations}


\section{Experimenting with Tactile Sensors}
\label{sec:3.2}
Since this project is primarily concerned with tactile data, the deployment and usage of DIGIT \cite{digit} tactile sensors was considered a top priority for the project.\\

To get familiar with the sensors before implementing our proposed learning framework, we experimented on a DIGIT tactile sensor and collected 200 examples of tactile readings manually. This dataset consisted of raw, colored, tactile readings of random finger positions and forces on the tactile sensors. Each example in the dataset is a 3-dimensional Numpy array with dimensions (240,320,3) which represents a RGB image. A simple figure below visualises 10 random examples of said tactile readings:
\begin{figure}[H]
    \centering
    \includegraphics[scale=0.4]{docs/Project Report/Media/tacto_testing.png}
    \caption{DIGIT tactile readings of 10 random examples}
    \label{fig:digit\_readings}
\end{figure}


\section{Pybullet Simulation}
\label{sec:3.3}
The fundamental features of the Pybullet simulation for this project (\hyperref[fig:pbSimScreenshot]{Figure 3.2} was developed collaboratively with Jeffery Wei, a Master's student working on a similar robotics project. This includes 1) combining URDF files for the real robot setup (arm, gripper and DIGIT sensors - see \hyperref[sec:3.1]{Section 3.1}), and 2) implementing three essential functions to support simple grasp planning, grasp execution, and manipulation of end effector poses using user-defined parameters.

\begin{figure}[H]
    \centering
    \includegraphics[scale=0.25]{docs/Project Report/Media/pybullet_sim.png}
    \caption{Screenshot of Pybullet simulation for the project}
    \label{fig:pbSimScreenshot}
\end{figure}

These parameters are defined through the Pybullet API, could be manually adjusted 


\subsection{Data Collection}
\label{sec:3.3.1}



%----------------------------------------------------------------------------------
% Section 4: Designing a Baseline Model and Deciding on Feature Representation of Tactile and Visual Data
%----------------------------------------------------------------------------------
\chapter{Baseline and Feature Representation Analysis}
\label{chap:4}
Mention the following:
- What common feature representations are there
- Results from your own testing
    - Collected tactile sensor readings (as tactile data) and end effector poses (as visual data)
    - Data was processed in 3 ways: raw, PCA with 3 main components, Convnet
    - Each type of processed data was then run with different feature combinations: tactile, vision \& tactile+vision
    - Display results
- Select the most promising representation and use it for actual generative model in Sec 4.


%----------------------------------------------------------------------------------
% Section 5: Actual Model
%----------------------------------------------------------------------------------
\chapter{Actual Model}
\label{chap:5}



%----------------------------------------------------------------------------------
% Section 5: Future Work
%----------------------------------------------------------------------------------
\chapter{Future Work}
\label{chap:6}
Since Pybullet is no longer being maintained, consider to move to mujico



%----------------------------------------------------------------------------------
% Appendix section
%----------------------------------------------------------------------------------
\appendix


\begin{thebibliography}{9}
% Academic papers
\bibitem{haarnoja}
    Haarnoja, T., Zhou, A., Abbeel, P. & Levine, S..
    \textit{Soft Actor-CriticL Off-Policy Maximum Entroy Deep Reinforcement Learning with a Stochastic Actor, \href{https://arxiv.org/abs/1801.01290}{https://arxiv.org/abs/1801.01290}},
    ICML,
    2018 roboti

\bibitem{nogueria}
    Nogueria et al,
    \textit{Unscented Bayesian Optimization for Safe Robot Grasping},
    IROS,
    2016

\bibitem{danielczuk}
    Danielczuk et al,
    \textit{Exploratory Grasping: Asymptotically Optimal Algorithms for Grasping Challenging Polyhedral Objects},
    2020

\bibitem{bekiroglu}
    Bekiroglu et al,
    \textit{Assessing Grasp Stability from Haptic Data},
    IEEE TRO,
    2011

\bibitem{frazier}
    Frazier,
    \textit{A Tutorial on Bayesian Optimization},
    2018

\bibitem{breyer}
    Breyer,
    \textit{\href{https://github.com/ethz-asl/vgn}{https://github.com/ethz-asl/vgn}}
    CORL,
    2020

% Academic papers for survey
\bibitem{calandra}
    Calandra et al,
    \textit{More Than a Feeling: Learning to Grasp and Regrasp using Vision and Touch},
    2018

\bibitem{hausman}
    Hausman et al,
    \textit{Generalizing Regrasping with Supervised Policy Learning},
    2017

\bibitem{farias}
    Farias et al,
    \textit{Simultaneous Tactile Exploration and Grasp Refinement for Unknown Objects},
    2021

% References for technologies used
\bibitem{digit}
    \href{https://digit.ml/}{DIGIT tactile sensors, 2020 @ Facebook}

\bibitem{pybullet}
    \href{https://pybullet.org/wordpress/}{Pybullet}

@article{Wang2022TACTO,
  author   = {Wang, Shaoxiong and Lambeta, Mike and Chou, Po-Wei and Calandra, Roberto},
  title    = {{TACTO}: A Fast, Flexible, and Open-source Simulator for High-resolution Vision-based Tactile Sensors},
  journal  = {IEEE Robotics and Automation Letters (RA-L)},
  year     = {2022},
  volume   = {7},
  number   = {2},
  pages    = {3930--3937},
  issn     = {2377-3766},
  doi      = {10.1109/LRA.2022.3146945},
  url      = {https://arxiv.org/abs/2012.08456},
}

\bibitem{franka}
    \href{https://www.franka.de}{Franka Emika}

\bibitem{robotiq}
    \href{https://robotiq.com/products/2f85-140-adaptive-robot-gripper}{Robotiq}

\end{thebibliography}

\chapter{Other appendices, e.g., code listing}
Put your appendix sections here

\end{document}